\chapter{METODOLOGIA}
\label{chap:metodologia}

Aspirando solucionar o problema de pesquisa levantado e detalhar o caminho a ser percorrido durante a pesquisa exploratória, para alcançar o objetivo do presente trabalho. A fim de delineação de pesquisa foi considerado as fases de levantamento e seleção da bibliografia, coleta de dados, critica dos dados, leitura analítica e fichamento das fontes, obtenção dos dados no sistema Pacificados, argumentação e discussão dos resultados.

As revisões de literaturas, sob o termo de interesse, será realizado uma busca, em Português e Inglês, nas seguintes fontes de busca:
\begin{itemize}
    \item Artigos  disponíveis  nos  anais do International  Command  and  Control  Research and  Technology  Symposium(ICCRTS),  organizado  anualmente  pelo  Departamento  de Defesa Americano;
    \item Artigos  científicos  indexados  no  Portal  de  Periódicos  da  Coordenação  de Aperfeiçoamento de Pessoal de Nível Superior (CAPES);
    \item Documentação  de  grupos  de  trabalho  relacionados  da  OTAN(Organização  do Tratado do Atlântico Norte);
    \item Documentação  de  grupos  de  trabalho  relacionados  da  OTAN(Organização  do Tratado do Atlântico Norte);
    \item Manuais de Campanha.
\end{itemize}

O presente estudo identifica-se como uma pesquisa do tipo exploratória com vista a avaliar o desempenho do sistema de comando e controle \textit{Pacificador}, nas operações. 