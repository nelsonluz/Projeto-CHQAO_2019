
\chapter{REFERENCIAIS TEÓRICOS}
\label{chap:refenciaisteoricos}

Segundo \citeonline{comandoecontrole2015} com o expresso uso de tecnologia nos combates modernos, devido a presença de civis, meios de difusão de informações no ambiente operacional. O outro fator é a necessidade de um  sistema de combate com maior proteção coletiva, velocidade e letalidade seletiva. Para tal fim, as ações de comando se desenrolam em todos seus estágios, até o cumprimento das mesma. O comandante, dentro desta linha, poderá consignar novas ordens, para melhor evolução da situação em determinada operação.

Para \citeonline{undestanding2006} nos dias de hoje as missões são ao mesmo tempo mais complexas e dinâmicas, onde requerem maior capacidade coletiva e organização muito mais efetiva para um maior sucesso operacional. O avanço da  Tecnologia de Informação vem criando um novo espaço onde organizações e indivíduos possam operar aprendendo a tomar vantagens sobre as oportunidades oferecidas nas operações sub-saindo aqueles que as ignoraram.


Ainda para \citeonline{comandoecontrole2015} O procedimento decisório envolve a obtenção de dados, a reunião de fatores intervenientes, o conhecimento da consciência situacional, até a decisão final. Nessa perspectiva, a ação de Comando e Controle (C2) é relevante ao triunfo nas diversas missões operacionais. Sua concepção se dá através de métodos, procedimentos, características  e vocabulários pecuniários de forma sistêmica. Nos dias atuais as decisões, dentro deste ambiente operacional, ficaram cada vez mais submetidas aos Sistemas de Tecnologia da Informação e Comunicações (TIC) no auxilio das execuções de comando e controle de forma rápida, efetiva e precisa. Os combates modernos têm nos sistemas de TIC todas as suas atividades operacionais e de apoio auxiliando o comandante, nos diversos níveis, suas decisões, com maior rapidez, análise, difusão dos conhecimentos para todos os escalões. Isto tudo busca uma melhor atividade operacional de comando e controle na incumbência da vitória. 



