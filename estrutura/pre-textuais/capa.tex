% Instituto Federal de Educação, Ciência e Tecnologia Baiano - Campus Guanambi
% 
% Modelo para Trabalho de Conclusão de Curso em LaTeX
% Superior de Análise e Desenvolvimento de Sistemas
% Alterado por: Dr. Naidson Clayr Santos Ferreira and Msc. Tiago Nogueira
%
% ----------------------------------------------------------------------- %
% Arquivo: capa.tex
% ----------------------------------------------------------------------- %

% CAPA---------------------------------------------------------------------------------------------------

% ORIENTAÇÕES GERAIS-------------------------------------------------------------------------------------
% Caso algum dos campos não se aplique ao seu trabalho, como por exemplo,
% se não houve coorientador, apenas deixe vazio.
% Exemplos: 
%\coorientador{Coloque o Nome do Coorientador}
%\departamento{Instituição do Coorientador}

% DADOS DO TRABALHO--------------------------------------------------------------------------------------
\titulo{Análise de um Sistema de Comando e Controle no apoio a decisão}
\titleabstract{Educational Data Mining for Performance Forecasting and Student Evasion in Introductory Algorithm Disciplines}
\autor{José Gustavo Sousa Peres e Nelson dos Santos Luz}
\autorcitacao{PERES, José, LUZ, Nelson} % Sobrenome em maiúsculo
\local{Brasília}
\data{2019}

% NATUREZA DO TRABALHO-----------------------------------------------------------------------------------
% Opções: 
% - Trabalho de Conclusão de Curso (se for Graduação)
% - Dissertação (se for Mestrado)
% - Tese (se for Doutorado)
% - Projeto de Qualificação (se for Mestrado ou Doutorado)
\projeto{Projeto do Trabalho de Conclusão de Curso}

% TÍTULO ACADÊMICO---------------------------------------------------------------------------------------
% Opções:
% - Bacharel ou Tecnólogo (Se a natureza for Trabalho de Conclusão de Curso)
% - Mestre (Se a natureza for Dissertação)
% - Doutor (Se a natureza for Tese)
% - Mestre ou Doutor (Se a natureza for Projeto de Qualificação)
\tituloAcademico{Tecnólogo em Gestão Militar}

% ÁREA DE CONCENTRAÇÃO E LINHA DE PESQUISA---------------------------------------------------------------
% Se a natureza for Trabalho de Conclusão de Curso, deixe ambos os campos vazios
% Se for programa de Pós-graduação, indique a área de concentração e a linha de pesquisa
\areaconcentracao{Análise e Desenvolvimento de Sistemas}
\linhapesquisa{Comando e Controle}

% DADOS DA INSTITUIÇÃO-----------------------------------------------------------------------------------
% Se a natureza for Trabalho de Conclusão de Curso, coloque o nome do curso de graduação em "programa"
% Formato para o logo da Instituição: \logoinstituicao{<escala>}{<caminho/nome do arquivo>}
\mec{Exército Brasileiro}
\setec{DECEx - DETMil}
\instituicao{Escola de Instrução Especializada}
%\departamento{Coint - Tecnologia em Sistemas para Internet}
\programa{Curso de Habilitação ao Quadro Auxiliar de Oficiais}
% \logoinstituicao{4cm}{Figuras/logoIsie.png} 

% DADOS DOS ORIENTADORES---------------------------------------------------------------------------------
\orientador{}
%\orientador[Orientadora:]{Nome da orientadora}
\instOrientador{Escola de Instrução Especializada}

\coorientador{}
%\coorientador[Coorientadora:]{Nome da coorientadora}
\instCoorientador{}
