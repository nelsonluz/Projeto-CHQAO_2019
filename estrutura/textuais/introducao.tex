% Instituto Federal de Educação, Ciência e Tecnologia Baiano - Campus Guanambi
% 
% Modelo para Trabalho de Conclusão de Curso em LaTeX
% Superior de Análise e Desenvolvimento de Sistemas
% Alterado por: Dr. Naidson Clayr Santos Ferreira
%
% ----------------------------------------------------------------------- %
% Arquivo: introducao.tex
% ----------------------------------------------------------------------- %

% INTRODUÇÃO-------------------------------------------------------------------

\chapter{INTRODUÇÃO}
\label{chap:introducao}
Devido ao permanente estado de prontidão do Exército Brasileiro (EB), para atender as demandas da defesa nacional buscando a garantia da lei e da ordem, desenvolvimento nacional e o bem-estar social. Segundo \citeonline{operacao2017}, com o advento das mudanças na sociedade e a crescente utilização de  tecnologia nas operações militares, as guerras têm experimentado alterações significativas ao longo dos tempos. Com as atuais estruturas geopolíticas na história dos conflitos temos o aumento da importância dos fatos militares e não militares na busca da resolução das conflagrações, através de recentes capacidades.

Hoje mesmo com o advento tecnológico, nas suas diversas áreas, os conflitos convencionais, bem como as disputas de alta intensidades mantém suas condutas predominantes. Nos dias atuais o processo de decisão, sob uma ação de comando, deve ser realizada no momento adequado e com a maior brevidade possível. Neste contexto, é que os sistemas de Comando de Controle, em particular o Pacificador, tem a finalidade de ser uma fonte de apoio a decisão no conjunto das operações militares dentro de um plano ou ordem intentando cumprir uma atividade, tarefa, missão ou atribuição. 

O objetivo deste trabalho, é realizar uma análise estatística do uso do Pacificador, um sistema de comando e controle desenvolvido pelo Centro de Desenvolvimento de Sistemas e usado por várias Organizações Militares (OM) do EB. O escopo desta análise será discorrer sobre quais são os incidentes mais comuns, média de tempo de resolução dos incidentes ao longo dos anos, quantas e quais as operações o sistema já apoiou, quantos usuários já utilizaram, quais e quantos Centros de Operações foram criados.


A metodologia encontra-se dividida em três fases onde, na primeira fase, são coletados os dados do sistema, desde o momento de sua colocação em produção. Na segunda fase são analisados quais os dados poderão ser melhor comparados para que sejam geradas informações relevantes à pesquisa e, por fim, na terceira fase, são demonstradas as melhorias que o sistema trouxe à Força Terrestre e as possíveis melhorias que podem ser implementadas.

A execução da abordagem metodológica, apresentada nesta seção, servirá como forma de trabalhar os conceitos estudados no referencial teórico, bem como responder às questões levantadas durante a apresentação do problema e justificativa da pesquisa (seção 2), além de alcançar os objetivos do trabalho, apresentados na seção 3.


