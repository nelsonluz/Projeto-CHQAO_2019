% Instituto Federal de Educação, Ciência e Tecnologia Baiano - Campus Guanambi
% 
% Modelo para Trabalho de Conclusão de Curso em LaTeX
% Superior de Análise e Desenvolvimento de Sistemas
% Alterado por: Dr. Naidson Clayr Santos Ferreira
%
% ----------------------------------------------------------------------- %
% Arquivo: introducao.tex
% ----------------------------------------------------------------------- %

% INTRODUÇÃO-------------------------------------------------------------------

\chapter{INTRODUÇÃO}
\label{chap:introducao}
Nos cursos superiores voltados às áreas da informática há um constante problema de evasão e baixo desempenho nas disciplinas introdutórias de programação.  \citeonline{gomes2008proposta} e \citeonline{manhaes2011previsao}, entre outros autores, descrevem que o problema vem da incapacidade de compreensão, por parte dos alunos, dos conceitos ligados às disciplinas e sua aplicação prática aos problemas cotidianos ou, até mesmo, do desbalanceamento das turmas principalmente quando envolvem grande quantidade de alunos por sala. No entanto, é impossível especular as causas que geram tal problema sem que antes haja uma análise aplicada diretamente às suas possíveis fontes. 

Para que seja possível que se tenha uma noção das principais causas de tal obstáculo na educação superior em determinadas instituições brasileiras é necessário que um estudo seja realizado possuindo um foco local, uma vez que estas representam, acima de tudo, um problema relacionado ao ambiente de convívio dos próprios estudantes e sua urgência de resolução demonstra-se indispensável. Este trabalho busca obter algumas respostas ao problema de baixo desempenho nos níveis educacionais, bem com predizer os ricos de evasão utilizando-se de técnicas de mineração de dados educacionais.

O objetivo deste trabalho, além de examinar as principais fontes causadoras da evasão e reprovação escolar, objetiva-se, também, à análise dos principais fatores que determinam o desempenho de alunos nas disciplinas introdutórias de programação, como algoritmos e estruturas de dados, em cursos superiores de informática para que seja possível identificar suas origens no ambiente pedagógico. Espera-se, com isso, a obtenção de resultados que, além demonstrar as causas das questões propostas, levantem outras questões de como nosso sistema educacional precisa ser trabalhado para que o potencial de inúmeros estudantes não deixe de ser aproveitado pela própria sociedade num futuro próximo.

A metodologia encontra-se dividida em três fases onde, na primeira fase, são apontados a instituição de ensino superior onde os dados serão trabalhados. Na segunda fase são descritos os procedimentos relacionados à coleta de dados, o local de onde as informações serão extraídas e os principais indicadores com os quais serão trabalhados e, por fim, na terceira fase, são demonstradas as abordagens e ferramenta que serão utilizadas durante o processamento dos dados coletados na segunda fase.

A execução da abordagem metodológica, apresentada na seção 5, servirá como forma de trabalhar os conceitos estudados no referencial teórico (seção 4), bem como responder às questões levantadas durante a apresentação do problema e justificativa da pesquisa (seção 2), além de alcançar os objetivos do trabalho, apresentados na seção 3.


