% Instituto Federal de Educação, Ciência e Tecnologia Baiano - Campus Guanambi
% 
% Modelo para Trabalho de Conclusão de Curso em LaTeX
% Superior de Análise e Desenvolvimento de Sistemas
% Alterado por: Dr. Naidson Clayr Santos Ferreira
%
% ----------------------------------------------------------------------- %
% Arquivo: Objetivo Geral.tex
% ----------------------------------------------------------------------- %

% OBJETIVOS-------------------------------------------------------------------

\chapter{OBJETIVOS}
\label{chap:objetivos}

\section{Objetivo Geral}
\label{sec:objetivogeral}

Realizar o levantamento qualitativo e quantitativo dos dados mais relevantes que sejam atinentes à parte operacional do Pacificador para tentar entender quais os incidentes mais comuns, média de tempo de resolução dos incidentes ao longo dos anos, quantas e quais operações o sistema já apoio, quantidade de usuários que utilizaram o sistema e quais os Centros de Operações foram criados ao longo dos anos de existência do Pacificador.
%Identificar, através da mineração de dados, as possíveis causas de reprovação e evasão, bem como analisar os principais fatores que  determinam o desempenho de alunos em disciplinas de algoritmo e programação nos cursos superiores e técnicos de informática, para que seja possível examinar suas principais origens dentro do contexto educacional.  

\section{Objetivos Específicos}
\label{sec:objetivosespecificos}

\begin{itemize}
    \item Indicar métodos utilizados na coleta de dados que possibilitem a análise qualitativa e quantitativa  da ferramenta;
    \item Identificar as causas mais comuns que dificultam e atrasam a resolução dos incidentes;
    \item Levantar quantas e em quais Operações o Pacificador já foi utilizado;
    \item Sugerir quais melhorias podem ser realizadas para refinar mais a utilização por parte dos usuários melhorando assim a resposta para os tomadores de decisão.
\end{itemize}