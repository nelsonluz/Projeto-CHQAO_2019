% Instituto Federal de Educação, Ciência e Tecnologia Baiano - Campus Guanambi
% 
% Modelo para Trabalho de Conclusão de Curso em LaTeX
% Superior de Análise e Desenvolvimento de Sistemas
% Alterado por: Dr. Naidson Clayr Santos Ferreira
%
% ----------------------------------------------------------------------- %
% Arquivo: definicao-problema.tex
% ----------------------------------------------------------------------- %

% DEFINIÇÃO DO PROBLEMA--------------------------------------------------------

\chapter{DEFINIÇÃO DO PROBLEMA}
\label{chap:definicaoproblema}

No ano de 2012 aconteceu no Brasil a Rio+20, Conferência das Nações Unidas sob o Meio Ambiente. Este seria um grande evento onde a questão segurança foi delegada ao Exército Brasileiro, no tocante a coordenação, preparo e emprego das Forças Federais, Estaduais e Municipais. Com este ambiente operacional foi criado um sistema de Comando e Controle, com vista a auxiliar os comandantes na tarefa de realizar a gerencia ações, para que a Conferencia ocorresse da melhor forma possível. Até os dias atuais este sistema vêm sendo utilizados nos diversos Comando de Área, buscando ser uma ferramenta de auxílio nas demandas operacionais. 

Dentro deste contexto operacional e das complexidades do teatro de operações, como o sistema de comando e controle Pacificador auxilia no apoio a decisão. Portanto, um estudo aprofundado no quesito é necessário para que sejam analisadas as utilizações do sistema \textit{Pacificador}, no auxilio aos comandantes nas diversas operações em todo Territórios Nacional, seja nos grandes eventos (FIFA 2014, Olimpíadas...) ou nas operações de GLO, até os dias atuais.

% as dificuldades envolvidas no ensino de programação em certas instituições, por se tratar de um problema de interesse comum, o que poderá, até mesmo, contribuir com processos que englobam temas socioeconômicos regionais inseridos na realidade de cada instituição, além de apresentar relevante valor para o meio acadêmico.


% O Problema de pesquisa é um questão específica que você quer investigar dentro do seu tema. Uma questão que pode e mereça ser investigada.

% Se o seu tema é logística, por exemplo, para desenvolver o seu TCC, você tem que definir uma questão específica que você quer investigar dentro da logística, do tipo: gestão de estoques e competitividade, custos de transportes… e por aí vai.

% E isso vale para qualquer área do conhecimento. Se o seu tema é Direito do Trabalho, para fazer um TCC, você precisa definir que questão específica quer investigar dentro das possibilidades do Direito do Trabalho. Talvez um ordenamento jurídico específico, por exemplo. Se o seu tema é educação infantil, você também precisa escolher uma questão específica que quer investigar nessa área. E isso vale para Medicina, Serviço Social, Odontologia… e para qualquer área de conhecimento.

